\documentclass{beamer}
\usepackage{polski}
\usepackage[utf8x]{inputenc}
\usepackage{graphicx}
\usepackage{hyperref}

\usetheme[hideothersubsections]{Hannover}

\title[PIC w~Pythonie]{Implementacja i~analiza wydajności programu do symulacji Particle-in\dywiz Cell w~języku Python}
\institute{Politechnika Warszawska}
\author{Dominik Stańczak, dr~Sławomir Jabłoński, dr~inż. Daniel Kikoła}
\date{TODO stycznia 2018}

\newcommand {\framedgraphic}[1] {
    % \begin{frame}
        \begin{center}
            \includegraphics[width=\textwidth,height=0.8\textheight,keepaspectratio]{#1}
        \end{center}
    % \end{frame}
}

\begin{document}
    \frame{\titlepage}
    \section{Fizyka plazmy}
        \begin{frame}[t]{Czym jest fizyka plazmy?}
            wew
        \end{frame}
    \section{Algorytmy Particle-in-Cell}
        \begin{frame}
            lol
        \end{frame}

        \begin{frame}[t]{Pętla obliczeniowa}

        \end{frame}

        \begin{frame}[t]{Interpolacja pól}

        \end{frame}

        \begin{frame}[t]{Poruszenie cząstkami}

        \end{frame}

        \begin{frame}[t]{Depozycja prądu i ładunku}

        \end{frame}

        \begin{frame}[t]{``Poruszenie polami'' - równania Maxwella}

        \end{frame}
    \section{Kod w Pythonie}
        \subsection{Wybrane rozwiązania}
            \begin{frame}

            \end{frame}
    \section{Profilowanie i kod w C++}
    \section{Wnioski}
\end{document}
