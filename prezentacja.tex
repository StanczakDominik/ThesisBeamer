\documentclass{beamer}
\usepackage{polski}
\usepackage[utf8x]{inputenc}
\usepackage{graphicx}
\usepackage{hyperref}
\usepackage[outdir=./]{epstopdf}
\usepackage{svg}
\usepackage{caption}


\usetheme[hideothersubsections]{Hannover}

\newcommand{\source}[1]{\caption*{Source: {#1}} }

\title[Particle-in-cell]{Algorytm particle-in-cell symulacji plazmy}
\author{Dominik Stańczak}
\date{5 kwietnia 2019}

\newcommand {\graphic}[1] {
    % \begin{frame}
        \begin{center}
            \includegraphics[width=\textwidth,height=0.8\textheight,keepaspectratio]{#1}
        \end{center}
    % \end{frame}
}


\newcommand {\framedgraphic}[1] {
    \begin{frame}
        \begin{center}
            \includegraphics[width=\textwidth,height=0.6\textheight,keepaspectratio]{#1}
        \end{center}
    \end{frame}
}

\AtBeginSection[]{
  \begin{frame}
  \vfill
  \centering
  \begin{beamercolorbox}[sep=8pt,center,shadow=true,rounded=true]{title}
    \usebeamerfont{title}\insertsectionhead\par%
  \end{beamercolorbox}
  \vfill
  \end{frame}
}

\begin{document}
    \frame{\titlepage}
    \section{Plazma}
        \begin{frame}[t]{Plazma}
          \graphic{plasma_uses}
        \end{frame}
        \begin{frame}[t]{Modelowanie plazmy}
          \graphic{methods}
        \end{frame}
    \section{Algorytm Particle-in-Cell}
        \begin{frame}[t]{Makrocząstki}
          \begin{itemize}[<+->]
            \item Skalowanie równań ruchu $\frac{q}{m} \to \frac{qN}{mN}$
            \item Możemy zastąpić $10^{23}$ cząstek przez $\approx 10^{6}$ \textbf{makrocząstek} po $\approx 10^{17}$ każda
            \item Dynamika się nie zmieni (acz statystyka może - szum!)
          \end{itemize}
        \end{frame}

        \begin{frame}[t]{Dyskretna siatka dla wielkości makroskopowych}
          \begin{itemize}[<+->]
            \item Zamiast na same cząstki, patrzymy na ich prąd i ładunek \textbf{zrzutowane} na dyskretną Eulerowską siatkę
            \item Brak liczenia sił międzycząstkowych $N^2$
            \item "Łagodzimy" dynamikę - tracimy biegun w $\frac{1}{r}$
            \item Ekranowanie $V(r) \sim \exp(-r) / r$ pojawia się samoistnie
          \end{itemize}
        \end{frame}
        
    \section{Pętla obliczeniowa PIC}

        \subsection{Push}
        \begin{frame}[t]{Push}
          \includesvg[width=\textwidth, height=0.8\textheight]{img/push.svg}
        \end{frame}

        \subsection{Scatter}
        \begin{frame}[t]{Scatter}
          \graphic{img/shapefunctions-eps-converted-to.pdf}
        \end{frame}

        \subsection{Solve}
        \begin{frame}[t]{\textbf{Solve} the Maxwell equations}
          \includesvg[width=\textwidth, height=0.4\textheight]{img/FDTD_Yee_grid_2d-3d.svg}
          {Siatka Yee używana do numerycznego rozwiązania elektromagnetycznych równań Maxwella. \\
          \url{https://commons.wikimedia.org/wiki/File:FDTD_Yee_grid_2d-3d.svg}}
        \end{frame}

        \subsection{Gather}
        \begin{frame}[t]{Gather}
          \graphic{img/shapefunctions-eps-converted-to.pdf}
          ... tylko w drugą stronę!
        \end{frame}
        % zaletą algorytmu jest rym

        \subsection{Gather}
        \begin{frame}[t]{Gather}
          \graphic{img/shapefunctions-eps-converted-to.pdf}
          ... tylko w drugą stronę!
        \end{frame}

        \begin{frame}[t]{\textbf{Pętla} obliczeniowa}
          \includesvg[width=\textwidth, height=0.8\textheight]{img/pic-cycle.svg}
        \end{frame}

    \section{Dwa strumienie}

        \begin{frame}[t]{Przypadek niestabilny - animacja}

        \end{frame}

        \begin{frame}[t]{Przypadek niestabilny}
          \graphic{img/TS_UNSTABLE_LARGE_DETAILS.pdf}
        \end{frame}

        \section{Elektromagnetyczne oddziaływanie lasera z tarczą}
        \begin{frame}[t]{Elektromagnetyczne oddziaływanie lasera z tarczą - animacja}
        \end{frame}
        \begin{frame}[t]{Pośrednie grzanie wodoru}
          \graphic{img/FULLLASER_DETAILS.pdf}
        \end{frame}

    \section{Wyniki - optymalizacja}
        \begin{frame}[t]{Skalowanie dla kodu w Pythonie}
          \graphic{img/speed_Python}
        \end{frame}

        \begin{frame}[t]{Skalowanie porównane z analogicznym kodem w C}
          \graphic{img/speed_Python_C}
        \end{frame}

        \begin{frame}[t]{Skalowanie porównane analogicznym z kodem w C zoptymalizowanym z flagą -O}
          \graphic{img/speed_Python_C_C-O}
        \end{frame}
        \begin{frame}[t]{GitHub}
          \begin{center}
          \large \url{https://github.com/StanczakDominik/pythonpic}
        \end{center}
          % free software - wolne oprogramowanie
        \end{frame}
    \begin{frame}[plain, c]
      \begin{center}
        \huge Bardzo dziękuję za uwagę!
      \end{center}
    \end{frame}
\end{document}
